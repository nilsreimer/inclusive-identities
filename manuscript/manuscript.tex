\documentclass[12pt, a4paper]{article}
\usepackage[utf8]{inputenc}

% Format
\setlength{\parindent}{0.5in}
\setcounter{secnumdepth}{0}
\usepackage[pagewise]{lineno}
%\usepackage{authblk}

% Font
\usepackage{MinionPro}
\input glyphtounicode
\pdfgentounicode=1
\usepackage{microtype}
\usepackage[super]{nth}

% Links
\usepackage[colorlinks=true, linkcolor=black, urlcolor=black, citecolor=black]{hyperref} 

% Language
\usepackage[british]{babel}

% References
\usepackage[nosectionbib, bibnewpage, tocbib, unnumberedbib]{apacite}

% Figures
\usepackage{graphicx}
\usepackage[small, labelfont=it, labelsep=period]{caption}

% Tables
\usepackage{booktabs}
\usepackage{tabularx}

% Commands
\newcommand{\pest}[4]{$ \text{Pr} (\text{``us''} | \text{#1}) = #2$, $[#3, #4]$}
\newcommand{\pdif}[4]{$ \Delta\text{Pr} (\text{``us''} | \text{#1}) = #2$, $[#3, #4]$}

% Frontmatter
\title{Intergroup contact fosters\\more inclusive social identities}
\date{July 30, 2019}

\begin{document}

\maketitle

\begin{abstract}
\noindent We examined how people construct their social identities from multiple group memberships---and whether intergroup contact can reduce prejudice by fostering more inclusive social identities. South Indian participants ($N = 351$) from diverse caste backgrounds viewed 24 identity cards, each representing a person with whom participants shared none, one, two, or all of three group memberships (caste, religion, nationality). Participants judged each person as ``us'' or ``not us'', showing whom they included in their ingroup, and whom they excluded. Participants tended to exclude caste and religious minorities, replicating persistent social divides. Bridging these divides, cross-group friendship was associated with more inclusive identities which, in turn, were associated with more favourable outgroup attitudes. Negative contact was associated with less inclusive identities, showing that past experiences shaped whom participants considered ``us'' or ``not us''. Contact and identity processes were unrelated to support for affirmative action in advantaged and disadvantaged caste groups.\\[1ex]
\noindent \textbf{Keywords:} intergroup contact, social identity, multiple categorization, intergroup relations, affirmative action \\[1ex]
\end{abstract}

%\linenumbers

\noindent How we feel about and act toward others depends on whom we consider ``us'' and ``them''---that is, whom we include in, and exclude from, our ingroup (for a review, see \citeNP{schmid_theories_2011}). In some situations, this distinction rests on one salient categorization, for example, someone's nationality at an international border. In diverse societies, however, this distinction often depends on multiple, overlapping group memberships. Individuals differ in how they construct their ingroup from these group memberships. Some espouse narrow definitions of who is ``us'' and ``them'', while others adopt more inclusive identities. Many Americans, for example, associate being American with being White American \cite{devos_american_2005}. This suggests that whom Americans consider ``us'' and ``them'' might depend on someone's race \emph{and} nationality. In this paper, we examine how people construct their social identities from multiple group memberships---and whether intergroup contact can reduce prejudice by fostering more inclusive social identities. We use a novel method to study social identification across multiple social categories and to examine possible antecedents (intergroup contact, social dominance orientation) and consequences (outgroup attitudes, intergroup threat, support for affirmative action) of more inclusive identities.

\begin{figure}
\centering
\includegraphics[scale=1]{../figures/figure-1}
\caption{
Schematic representations of social identity structures \protect\cite{roccas_social_2002}, ordered by their social identity inclusiveness \protect\cite{dommelen_construing_2015}. Shaded regions represent the groups which a participant has to categorize as ``us'' to be assigned that structure. An Indian Hindu, for example, might consider only people who share their nationality, religion, and caste as ingroup members (intersection). Someone else might consider all their fellow Indians, whatever their religion or caste, as ingroup members (dominance). Another person might consider anyone who shares their nationality or religion as ingroup members (merger).
}
\label{fig:f1}
\end{figure}

Social psychologists highlighted the importance of considering two \cite{berry_immigration_1997, crisp_multiple_2007, dovidio_commonality_2009} or more \cite{roccas_social_2002} social categories for understanding intergroup relations. \citeA{crisp_multiple_2007} reviewed evidence that we tend to have more favourable attitudes towards people with whom we share some, but not all, group memberships than towards people with whom we do not share any group memberships. \citeA{gaertner_reducing_2000} showed that espousing a more inclusive common ingroup identity (e.g., Indian) over a narrower identity (e.g., Hindu, Muslim) can extend the benefits of ingroup favouritism to outgroup members and thus reduce intergroup bias. \citeA{roccas_social_2002} recognized that, though we all hold multiple overlapping group memberships, we differ in how we construct our social identities from these group memberships. Broadly, these perspectives recognize that we are part of multiple overlapping groups; that we use these group memberships to construct a subjective sense of who is ``us'' and ``them''; and that espousing more inclusive social identities can reduce prejudice and discrimination.

Diverse societies confront their members with the decision of which intersections of various group memberships they include in their subjective ingroup. One Indian Hindu, for example, might constrict their ingroup to people of the same nationality, religion, and caste. Another Indian Hindu might consider all Indians, whatever their religion or caste, as ingroup members (see Figure~\ref{fig:f1}). Past research \cite{dommelen_construing_2015} indeed observed substantial individual differences in whom participants considered ``us'' and ``not us''. Some participants espoused more inclusive social identities---that is, they included more people in their subjective ingroup. Other participants espoused less inclusive social identities---that is, they included fewer people in their subjective ingroup. All participants, however, belonged to the same disadvantaged minority group. Past research thus provided evidence for \emph{individual differences}, but did not examine \emph{group differences} in whom people consider ``us'' and ``not us''.

Group membership, however, often shapes whom people include in, and exclude from, their subjective ingroup. For majority-group members, negotiating their ethnic and national identities means something different than for minority-group members (for a review, see \citeNP{dovidio_commonality_2009}). Majority-group members face the choice of including or excluding ethnic, religious, and other minorities from their subjective ingroup. Choosing to include or exclude minorities could, for example, reflect whether a majority-group member holds multicultural or nativist views. Minority-group members face the choice of aligning themselves with the dominant majority group, or seeking a common ingroup identity with other subordinate minority groups. Choosing an identity that includes either the majority group or other minority groups could, for example, reflect whether minority-group members seek to derogate or form a coalition with other minority groups \cite{craig_coalition_2012}. We expect that group differences are as great as, if not greater than, individual differences in social identity construals---especially if groups differ in status and power. Our research includes participants from advantaged, intermediate, and disadvantaged backgrounds to study how groups differ in whom their members consider ``us'' and ``not us''.

How we think and feel about others depends on whether we include them in our ingroup. Dividing others into ``us'' and ``not us'' is thus a necessary and sufficient condition for ingroup favouritism \cite{tajfel_human_1981}. Past research has shown that people favour ingroup members in resource allocation \cite{tajfel_social_1971}, in causal attribution \cite{hewstone_ultimate_1990, taylor_ethnocentrism_1974}, in helping behaviour \cite{levine_identity_2005}, in information processing \cite{bavel_neural_2008}, and in evaluation \cite{otten_evidence_2000}. Considering more people as ``us'' should extend ingroup favouritism to a wider range of people, and thus reduce prejudice and discrimination (for a review, see \citeNP{gaertner_common_2016}). Our research examines whether fostering more inclusive social identities leads to more favourable attitudes and less social distance toward national, religious, and caste outgroups.

Contact with members of other groups changes how we think and feel about them. Past research has shown that intergroup contact reduces prejudice \cite{pettigrew_meta_2006} by decreasing intergroup anxiety, increasing empathy and perspective-taking, and by increasing knowledge about the other group \cite{pettigrew_how_2008}. In addition, researchers have proposed that contact could reduce prejudice by changing how we see ourselves
\cite{pettigrew_generalized_1997, pettigrew_intergroup_1998}. Encountering diverse others could initiate a process of cognitive differentiation \cite{schmid_social_2011} whereby individuals become aware of complex interrelations between their group memberships and thus adopt more complex social identities \cite{schmid_antecedents_2009}---that is, they become aware that they belong to multiple categories and recognize that these multiple categories do not always overlap \cite{roccas_social_2002}. Other research has found that intergroup contact can, under the right conditions, motivate people to adopt a more inclusive common ingroup identity \cite{gaertner_contact_1994} and to include the relevant outgroup in their self-concept \cite{page_understanding_2010}. If intergroup contact motivates people to adopt social identities that include outgroup members, it should reduce prejudice and discrimination against them. Our research examines whether intergroup contact can reduce prejudice against religious and caste outgroups by fostering social identities that include them as ingroup members.

Beyond prejudice, researchers have debated whether, by fostering more inclusive identities, intergroup contact helps or hinders social change. Social change often requires advantaged and disadvantaged groups to strive for redistributive policies. On the one hand, fostering inclusive identities could break down the boundaries between the advantaged ingroup and the disadvantaged outgroup---turning injustice faced by ``them'' into ``our'' problem. This process could narrow the gap between advantaged-group members' support for the principle of equality and their opposition to its implementation \cite{dixon_intergroup_2007}. On the other hand, fostering more inclusive identities could distract disadvantaged-group members from differences in resources and power, and thus reduce their demands for and support for action aimed at social change \cite{dovidio_darker_2012, saguy_irony_2009}. Together, these mechanisms suggest that fostering more inclusive identities could increase support for social change among the advantaged, but decrease support among the disadvantaged. Our research examines how intergroup contact, by fostering more inclusive social identities, affects support for affirmative action policies, as well as perceptions of relative advantage and disadvantage, in advantaged and disadvantaged groups.

\subsection{Caste, Religion, Nation in South India}

We investigated these questions in the context of caste, religion, and nationality in South India. South India provides an important context for studying social identification across multiple social categories. India is diverse. South Indians belong to different castes, religions, and other groups that converge, diverge, and intersect in interesting ways. India is also underrepresented in psychological research.\footnote{Among the non-WEIRD \cite{henrich_weirdest_2010} countries underrepresented in social psychological research, India is a prominent omission as it spans almost a fifth of the world's population.} Few studies \cite<e.g.,>{ghosh_hindu_1991, tausch_relationships_2009} of intergroup relations focus on religious relations in India. Even fewer \cite<e.g.,>{cotterill_ideological_2014} focus on caste relations. Our research examined the cross-cutting categories of caste, religion, and nationality in a diverse sample of South Indian students.

Caste is a system of social relations that concentrates resources in the hands of dominant castes by restricting subordinate castes' occupations and land ownership, and by enforcing endogamy and segregation \cite{jodhka_caste_2012}. This system is underpinned by a tradition that ranks castes according to their supposed ritual purity, and condemns contact with supposedly less pure castes. Since Independence, the Indian State has sought to redress caste-based injustices by enforcing `reservations' of seats in state-run universities and state-sector jobs for disadvantaged groups. It recognizes Dalits, members of castes affected by untouchability, as `Scheduled Castes' (SCs); Adivasi, India's indigenous peoples, as `Scheduled Tribes' (STs); and other disadvantaged groups, who occupy an intermediate position in the caste hierarchy, as `Other Backward Classes' (OBCs). Historically advantaged castes do not have access to reserved seats and jobs (GM: `General Merit'). Reservation policies, alongside ongoing discrimination, mean that caste identities remain important in electoral politics and political organizing.

South India is religiously diverse. In Karnataka, where we conducted this research, 84\% of inhabitants are Hindus, while 13\% are Muslims. Muslims face both structural inequalities and communal violence. In 2002, for example, over a thousand Muslims were killed in a pogrom in Gujarat \cite{dhattiwala_political_2012}. In recent years, anti-Muslim violence has been increasing \cite{amnesty_india_2017}. Religion is intertwined with two currents of Indian nationalism \cite{menski_hindu_2009}. On the one hand, the secular foundations of the Indian State resulted from an inclusive nationalism that strives to include Indians of all religions. On the other hand, Hindu nationalism (Hindutva) is an ideology that equates being Indian with being Hindu, thus excluding Muslims from the national identity. Narendra Modi's Bharatiya Janata Party (BJP) government espouses Hindutva and enjoys broad support in the Indian population \cite{pew_three_2017}. Indians thus face a choice between two ideologies that either include Indian Muslims in the national identity or exclude them from it.

\section{Present research}

We examined how people construct their social identities from multiple group memberships, whether intergroup contact is associated with more inclusive social identities, and how more inclusive social identities relate to prejudice and support for social change. Our research used a novel method to study social identification across multiple categories in a sample of South Indians from diverse caste backgrounds. Below, we first introduce this experimental task before reviewing the research questions and hypotheses guiding the present research.

We adapted the triple crossed-categorization task \cite{dommelen_construing_2015} to assess how participants construct their social identities from multiple group memberships. In this task, participants view identity cards, each representing a person with whom participants shared none, one, two, or all of three group memberships. Participants report whether they consider each person as ``us'' or ``not us'', showing whom they include in, or exclude from, their ingroup. This task allowed us to study identification across multiple social categories and how it relates to the predictors and outcomes in our hypotheses. Van Dommelen and colleagues \citeyear{dommelen_construing_2015} used a similar task to examine whom participants considered ``us'' and ``not us''. However, their research included participants from only one group, aggregated participants' categorizations into a single score, and tested whether that score correlated with aggregated responses to contact and attitude measures.\footnote{Van Dommelen et al. \citeyear{dommelen_construing_2015} summarized participants' responses qualitatively by categorizing them into distinct \emph{social identity structures} and quantitatively by counting the total number of ``us'' responses as \emph{social identity inclusiveness}.} Our research included participants from multiple groups and used multilevel modeling to test whether contact with and attitudes toward \emph{each} group were associated with the inclusion of \emph{that} group in one's social identity. This allowed for a more fine-grained analysis of individual and group differences in identification across multiple social categories.

We pursued four substantive research questions. First, we examined how participants' group memberships shaped whom they included in, and excluded from, their ingroup. We hypothesized that participants would exclude targets from religious and national outgroups---and that participants from advantaged castes would exclude targets from disadvantaged castes, and vice versa. We did not have a clear prediction for participants from intermediate caste backgrounds. Second, we examined whether past experiences with members of caste and religious outgroups shaped whom participants included in, and excluded from, their ingroup. We hypothesized that positive contact and cross-group friendship would be associated with more---and negative contact \cite{barlow_contact_2012, hayward_toward_2017} with less---inclusive social identities. Third, we examined whether, by fostering more inclusive social identities, intergroup contact would be associated with less prejudice. We hypothesized that categorizing someone as ``us'' would be associated with more favourable attitudes and less social distance toward that person. Fourth, we examined how intergroup contact, by fostering more inclusive social identities, affects support for affirmative action in advantaged and disadvantaged groups. We hypothesized that more inclusive social identities would be associated with less perceived discrimination and less support for affirmative action among the disadvantaged, and with more perceived privilege and more support for affirmative action among the advantaged.

We examined two other constructs that might be related to more and less inclusive social identities. First, we included ideological preferences for group dominance \cite{sidanius_social_1999} as an additional predictor variable to rule out a potential confound, whereby people high in social dominance orientation might both eschew contact with caste and religious minorities and exclude them from the common ingroup. Second, we included perceptions of intergroup threat as an additional outcome variable to test whether more inclusive social identities were associated with this more distal outcome. Perceptions of intergroup threat play an important role in shaping intergroup relations (for a review, see \citeNP{stephan_intergroup_2016}). We tested whether participants who included caste and religious minorities in their ingroup would, in addition to feeling more warmth and less social distance toward them, feel less threatened by them in terms of both realistic and symbolic threat.

\section{Method}

All materials, data, analysis scripts, and appendices are available online (\href{https://osf.io/ekb8z/?view_only=05b6a5c5cf5e43d9a7bba5e192f53f87}{https://osf.io/ ekb8z/?view\_only=05b6a5c5cf5e43d9a7bba5e192f53f87}). Here, we only report measures testing our hypotheses, omitting measures replicating earlier research or validating the categorization task (Appendix~F). Reanalysing existing data \cite{dommelen_construing_2015}, we determined that $\sim$100 respondents per group would allow reasonably precise estimates of model parameters.\footnote{Models estimated posterior probabilities with precision $\textit{SD} < 0.5$ on the log odds scale.} We report descriptive statistics and interscale correlations in Appendix~B. 

\subsection{Participants}

We recruited 351 students at Karnatak University (Dharwad, India). Of these, we excluded 49 participants who did not belong to any of four caste groups ($n = 20$), failed to indicate their caste group ($n = 7$), or reported Islam as their own or their family's religion ($n = 27$).\footnote{We excluded Muslim participants as this subsample was too small for meaningful analyses.} This left 302 participants who reported Hinduism ($n = 286$), Jainism ($n = 8$), or Christianity ($n = 8$) as their or their family's religion, and General Caste ($n = 99$), Other Backward Class (n = $127$), Scheduled Caste ($n = 54$), or Scheduled Tribe ($n = 22$) as their caste group. Table~\ref{tab:t1} summarizes participants by gender, age, nationality, religion, and caste. 

\begin{table}    
\caption[Participants by gender, age, nationality, religion, and caste]{Participants by gender, age, nationality, religion, and caste. Categories in \textit{italics} were excluded from the final sample. N/A marks missing responses.}
\centering
\figureversion{lining, tabular}
\small	
\begin{tabular}{llrr} \addlinespace \toprule
\multicolumn{2}{l}{Category} & $n$ & \% \\ \midrule \addlinespace 
Gender      & Woman      & 215 & 61 \\
            & Man & 121 & 34 \\
            & Other & 0 & 0 \\
            & N/A & 15 & 4 \\ \addlinespace \addlinespace
Age         & 18--20 & 1 & 0 \\
            & 21--23 & 254 & 72 \\
            & 24--26 &  77 & 22 \\
            & 27--29 &  10 &  3 \\
            & 30--32 &   1 &  0 \\
            & 33--35 &   0 &  0 \\
            & 36 or older & 1 & 0 \\ 
            & N/A & 7 & 2 \\ \addlinespace \addlinespace
Nationality & Indian & 339 & 97 \\
            & Other & 0 & 0 \\
            & N/A & 12 & 3 \\ \addlinespace \addlinespace
Religion    & Buddhism & 1 & 0 \\ 
            & Christianity & 11 & 3 \\ 
            & Hinduism & 297 & 85 \\ 
            & \textit{Islam} & \textit{27} & \textit{8} \\ 
            & Jainism & 8 & 2 \\ 
            & Other & 2 & 1 \\ 
            & N/A & 5 & 1 \\ \addlinespace \addlinespace
Caste       & General Caste & 104 & 30 \\ 
            & Other Backward Class & 143 & 41 \\ 
            & Scheduled Caste & 54 & 15 \\ 
            & Scheduled Tribe & 23 & 7 \\ 
            & \textit{Other / Not applicable} & \textit{20} & \textit{6} \\ 
            & \textit{N/A} & \textit{7} & \textit{2} \\ \addlinespace \midrule
Total       &   & 351 & 100 \\ \bottomrule
\end{tabular}
\label{tab:t1}
\end{table}

\subsection{Procedure}

Participants completed a triple crossed-categorization task, in which they viewed 24 identity cards. Each showed a fictitious person's name, age, religion, nationality, caste reservation, and a head-and-shoulders silhouette. Based on a pilot study, we manipulated the target's caste, religion, and nationality such that each target represented a person with whom participants shared none, one, two, or three group memberships (Figure~\ref{fig:f2}). We tested participants in classrooms of 24--71 students by presenting targets in a slide-based presentation and by handing each student a pen-and-paper questionnaire. Each slide contained a male and a female target with participants focusing on the target corresponding to their gender. Slides also contained a number identifying each target, and the response scale(s) corresponding to the question(s) participants answered at the time.

\begin{figure}
\centering
\includegraphics[scale=1]{../figures/figure-2}
\caption{Examples of targets used in the triple crossed-categorization task. Based on ratings in a pilot study ($N = 26$), we selected the four most prototypical targets (out of fifty initial targets) for each of six plausible combinations of caste, religion, and nationality (for details, see Appendix~A). Each target showed a person's caste reservation (GM = General Merit, OBC = Other Backward Class, SC/ST = Scheduled Caste/Scheduled Tribe), religion (Hindu, Muslim), and nationality (Indian, Nepali, Sri Lankan, Bangladeshi). Each target also showed the person's first and last name, age (21--26 years), and a silhouette corresponding to the person's gender (adapted from \protect\citeNP{ma_chicago_2015}). Each target's age and silhouette, as well as the order in which the targets were presented, varied across sessions.}
\label{fig:f2}
\end{figure}

In each session, participants first familiarized themselves with each target (shown for 7 s in a randomized order) in an automated slideshow. Participants viewed targets (in the same order) for a second time, noting for each target whether they felt that this person was one of their own group (1 = ``us''), or not one of their own group (0 = ``not us''). Participants viewed targets for a third time, rating how comfortable or uncomfortable they would feel to share a room with this person (social distance; $1 = \textit{very uncomfortable}$, $7 = \textit{very comfortable}$), and how they felt toward this person (feeling thermometer; $0 = \textit{cold}$, $100 = \textit{warm}$). Participants then completed the measures described below.

\subsection{Measures}

We measured intergroup contact as: how often, from $1 = \textit{never}$ to $5 = \textit{very often}$, participants meet outgroup members in their everyday life (contact quantity), and how often, on average, they have positive/good contact and negative/bad contact with outgroup members \cite{barlow_contact_2012}. We preceded these items with examples of positive and negative contact experiences. We measured cross-group friendship with two items \cite{turner_reducing_2007}: ``How many close friends do you have who are [outgroup members]?'' ($1 = \textit{none}$, $5 = \textit{more than ten}$), and ``How often do you spend time with [outgroup] friends?'' ($1 = \textit{never}$ to $5 = \textit{very often}$; $.47\leq\textit{rs}\leq.58$). Participants reported contact with four groups: Dalits, people from other backward classes, people from general castes, and Muslims.

We measured social dominance orientation as how much, between $1 = \textit{strongly oppose}$ and $7 = \textit{strongly favour}$, participants endorsed eight statements about social hierarchies \cite{ho_nature_2015}. Four items measured support for group-based dominance (SDO-Dominance), for example, ``some groups of people are simply inferior to other groups''. Four items measured opposition to egalitarian ideologies (SDO-Egalitarianism), for example, ``it is unjust to try to make groups equal''.\footnote{Replicating \citeauthor{ho_nature_2015}’s \citeyear{ho_nature_2015} findings, we found a four-factor model (with two method factors) to represent the data better than the one-factor ($\Delta\chi^2 = 155.14$, $p < .001$) or two-factor ($\Delta\chi^2 = 138.67$, $p < .001$) alternatives. Accounting for this structure, we used latent factor scores for SDO-Dominance and SDO-Egalitarianism in our analyses.}

We measured realistic threat---from Muslims and Dalits---with three items per outgroup \cite{schmid_reducing_2014}, for example, ``the more power [Muslims/Dalits] gain in this country, the more difficult it is for [Hindus/people from my caste group]'' ($1 = \textit{strongly disagree}$, $5 = \textit{strongly agree}$; $\alpha_\textit{Muslims} = .71$, $\alpha_\textit{Dalits} = .79$). Two items measured symbolic threat, e.g., ``[Muslims/ Dalits] threaten [Hindus'/my caste group's] way of life'' ($1 = \textit{strongly disagree}$, $5 = \textit{strongly agree}$; $r_\textit{Muslims} = .44$, $r_\textit{Dalits} = .48$).

We measured perceived (dis-)advantage as how easy or hard participants thought it was, on average, for people from various groups to succeed in India today ($1 = \textit{very hard}$, $7 = \textit{very easy}$). Participants rated seven groups: People from your own background, Scheduled Caste, Scheduled Tribe, Other Backward Class, General Caste, Hindus, and Muslims.

Five items measured policy support. Participants read that ``currently, 22.5\% of seats in central-government funded universities are reserved for Scheduled Caste and Scheduled Tribe students'', and that ``an additional 27.5\% of seats in central-government funded universities are reserved for students from Other Backward Classes''. Participants indicated to what extent they opposed or supported reservation in higher education for students from each group ($1 = \textit{strongly oppose}$, $5 = \textit{strongly support}$), and whether they thought that reservation in higher education for students from that group should increase, decrease, or remain unchanged ($1 = \textit{decrease a lot}$, $5 = \textit{increase a lot}$; $r_\textit{SC/ST} = .67$, $r_\textit{OBC} = .67$). Participants then read that ``no seats in central-government funded universities are reserved for Muslim students nationally, though some states have introduced quotas for Muslim students''. Participants indicated to what extent they opposed or supported reservation for Muslim students.

\section{Results}

We used the following analysis strategy: First, we estimated group differences in whom participants categorized as ``us'' and ``not us''. Second, we examined to what extent intergroup contact and ideological orientations explained individual differences in participants' categorizations. Third, we tested whether participants' categorizations were associated with their attitudes and beliefs.

\subsection{Group differences}

We examined \emph{how likely} participants were to categorize \emph{which} target as ``us'' versus ``not us''---and how that probability varied as a function of the participants' and the targets' group memberships (\emph{group differences}) and of the participants' past experiences and ideological orientations (\emph{individual differences}).\footnote{Van Dommelen et al. \citeyear{dommelen_construing_2015} analysed \emph{which} and \emph{how many} targets participants included in their ingroup as distinct questions; in Appendix~C, we report analyses using their operationalization.}

To answer these questions, we ran a series of multilevel logistic regression models with participants' target categorizations (1 = ``us'', 0 = ``not us'') as outcome variable. We estimated these models in RStan \cite{rstan_package} using Bayesian statistical methods. Bayesian inference involves choosing a likelihood function and prior distributions. The likelihood function links the observed data to one or more model parameters (e.g., regression coefficients) and states how likely the observed data are given different values of said model parameters. Prior distributions state how plausible different values of said model parameters are before considering the observed data. Our models derived the likelihood of the observed responses from a generalized linear model with a logit link function. Our models assigned weakly informative prior distributions to all model parameters \cite{gelman_prior_2017}.\footnote{Our models assigned the following priors: $\beta \sim \text{Student}(2.5, 0, 1)$ for fixed effects and $\sigma \sim \text{Half-Cauchy}(0, 1)$ for standard deviations of varying effects.}  Bayesian inference applies Bayes' theorem to update prior distributions in light of the observed data to produce posterior distributions. Other than $p$-values and confidence intervals, the resulting posterior distributions have a straight-forward interpretation as stating how plausible different values of the model parameters are given the observed data.\footnote{For a detailed introduction, consult \citeA{gelman_bayesian_2014}, \citeA{lambert_student_2018}, and	 \citeA{mcelreath_statistical_2016}.} 

To test our hypotheses, we compared a series of nested models. We used stratified 10-fold cross-validation to estimate how accurately each model predicts observations outside of the sample used to estimate said model \cite{vehtari_practical_2017}. We selected more complex over simpler models when the difference in predictive accuracy was at least twice its standard error.\footnote{Selecting models based on their out-of-sample predictive accuracy balances the risks of underfitting and overfitting \cite<see also>{yarkoni_choosing_2017}. Standard errors, in this case, come from treating the observed data as a sample from a population and, as such, have the same interpretation as standard errors in classical statistics \cite{vehtari_practical_2017}.} Model~0 estimated the probability of participants categorizing a target as ``us'' as a function of a fixed intercept and varying (random) intercepts for all participants.\footnote{Models used the used the non-centred parameterization to represent all varying effects \cite{betancourt_hamilton_2015}.} Model~1 estimated this probability as varying across participants and target categories. Models~2 and 3 estimated, respectively, how SC/ST participants' categorizations of Indian targets differed from GM and OBC participants' and how OBC participants' categorizations differed from GM participants'. Models~1 and 2, but not Model~3, made more accurate predictions than less complex models (Table~\ref{tab:t2}). This suggests that the targets' group memberships shaped participants' categorizations, and that GM and OBC participants' categorizations resembled each other but differed from SC/ST participants'. Below, we report and compare the estimated probabilities of ``us'' categorizations for different combinations of targets' and participants' group memberships. We report uncertainty intervals spanning the 97\% most plausible estimates \cite{coda_package}.

\begin{table}
\caption{
Comparison of models estimating the probability of participants categorizing targets as ``us'' versus ``not us''. $\textit{ELPD}$ is the expected log predictive density, with higher numbers indicating that a model is expected to make more accurate out-of-sample predictions \protect\cite{vehtari_practical_2017}. $\Delta\textit{ELPD}$ is the difference in $\textit{ELPD}$ between the current and previous model, with positive values indicating that the current model is expected to make more accurate out-of-sample predictions. We selected a more complex model over a simpler model when $\frac{\Delta\textit{ELPD}}{\textit{SE}} \geq 2$.%$w$ are stacking weights based on the models' expected log predictive densities \protect\cite{yao_using_2018}.
}
\centering
\figureversion{lining, tabular}
\small
\begin{tabularx}{\linewidth}{r@{~}rXrrrrr} \toprule
\# &  & Description & $\textit{ELPD}$ & $\textit{SE}$ & $\Delta\textit{ELPD}$ & $\textit{SE}$ & $\frac{\Delta\textit{ELPD}}{\textit{SE}}$ \\ \midrule 
0 &      & Intercept (Participant) & -4703.5 & 23.7 &     - &    - &    - \\ 
1 & vs 0 & Intercept (Category)    & -3853.5 & 41.6 & 850.1 & 37.4 & 22.7 \\
2 & vs 1 & Group differences (SC/ST)       & -3791.3 & 42.0 &  62.1 & 11.1 &  5.6 \\
3 & vs 2 & Group differences (OBC)         & -3792.1 & 42.2 &  -0.8 &  3.6 & -0.2 \\ \midrule
4 & vs 2 & Intergroup contact (4) & -3743.4 & 42.8 &  47.9 &  8.8 &  5.4 \\
5 & vs 4 & Intergroup contact (2) & -3736.7 & 42.6 &   6.7 &  1.8 &  3.8 \\
6 & vs 5 & Intergroup contact (2) & -3747.8 & 42.7 & -11.1 &  3.2 & -3.5 \\
7 & vs 2 & Social dominance orientation & -3791.0 & 42.1 &   0.4 &  3.3 &  0.1 \\
\bottomrule
\end{tabularx}
\label{tab:t2}
\end{table}

\begin{figure}
\centering
\includegraphics[scale=1]{../figures/figure-3}
\caption{
Estimated probability of participants categorizing a target as ``us'' versus ``not us'' by targets' nationality, religion, and caste (vertical), and participants' caste membership (horizontal). Dots (•) indicate the most plausible \emph{estimate} for a given target's probability of being included in participants' ingroup (in Model~2, Table~\ref{tab:t2}), while the shaded ribbons encompass the 67\% (darkest shade), 89\%, and 97\% (lightest shade) most plausible estimates of that probability. Pluses (+) indicate the \emph{observed} proportion of participants who included a given target in their ingroup. Comparing predicted and observed proportion shows that the model represents the data reasonably well. GM = General Merit, OBC = Other Backward Class, SC/ST = Scheduled Caste/Scheduled Tribe.
}
\label{fig:f3}
\end{figure}

Figure~\ref{fig:f3} shows the estimated probabilities of General Merit (GM), Other Backward Class (OBC), and Scheduled Caste/Scheduled Tribe (SC/ST) participants categorizing a target as ``us''. Participants tended to exclude foreigners from their subjective ingroup, though about half of the responses indicated more inclusive identities. Few participants considered Bangladeshi Muslims part of their ingroup, \pest{M2}{.17}{.14}{.21}, though roughly half of the participants included Sri Lankan and Nepali Hindus in their ingroup, \pest{M2}{.46}{.41}{.52}. Participants were thus more likely to include foreign targets when they were Hindu, \pdif{M2}{.29}{.25}{.34}. Still, GM/OBC and SC/ST participants were, respectively, $1.95$, $[1.73, 2.18]$ and $1.90$, $[1.71, 2.14]$ times more likely to categorize Indian, Hindu targets as ``us'' compared to foreign, Hindu targets.

As expected, participants' caste membership shaped how they categorized Indians of different castes and religions. Most GM and OBC participants included Hindu, GM and Hindu, OBC targets in their ingroup, \pest{M2}{.94}{.92}{.96} and \pest{M2}{.93}{.91}{.95}. Fewer GM and OBC participants categorized Hindu, SC/ST targets as ``us'', \pest{M2}{.84}{.80}{.87}. They were least likely to categorize Muslim, OBC targets as ``us'', \pest{M2}{.75}{.70}{.80}. As such, participants from advantaged (GM) and intermediate (OBC) caste backgrounds tended to exclude targets from disadvantaged caste groups (SC/ST) and ethnoreligious minorities (Muslims). SC/ST participants' responses differed from GM and OBC participants'. Almost all SC/ST participants included Hindu, SC/ST targets in their ingroup, \pest{M2}{.97}{.95}{.99}. Fewer SC/ST participants included Hindu, GM and Hindu, OBC targets in their ingroup, \pest{M2}{.88}{.83}{.92} and \pest{M2}{.80}{.73}{.86}, respectively. SC/ST participants were even less likely than others to include Muslim, OBC targets as ``us'', \pest{M2}{.47}{.38}{.57}. Overall, these findings showed that participants from advantaged backgrounds tended to exclude targets from disadvantaged backgrounds (and vice versa), while all participants tended to exclude targets from the Muslim minority.

\subsection{Individual differences}

Group memberships shaped whom participants considered ``us'' and ``not us'', though none of these associations were without exception. We examined to what extent individual differences in past experiences and ideological orientations explain why some participants excluded targets from caste and religious outgroups---and why others did not.

Models~4 to 6 tested whether intergroup contact was associated with how participants categorized Indian targets of caste or religious outgroups. Model~4 extended Model~2 by including contact quantity, positive contact, negative contact, and outgroup friendship as predictors of participants’ categorizations. Model~4 made more accurate predictions than Model~2. Negative contact ($e^\beta = 0.81$, $[0.72, 0.90]$) and outgroup friendship ($e^\beta = 1.50$, $[1.28, 1.72]$) were associated with participants' categorizations, but neither positive contact ($e^\beta = 1.01$, $[0.87, 1.16]$) nor contact quantity ($e^\beta = 0.99$, $[0.86, 1.15]$) were.\footnote{An odds ratio of $e^\beta = 1$ means that a predictor does not affect the odds of the relevant outcome; odds ratios of $e^\beta > 1$ and $e^\beta < 1$ mean, respectively, that a predictor makes the relevant outcome more and less likely.} Model~5 included only negative contact and outgroup friendship, and made more accurate predictions than Model~4. Model~6 estimated the relationships between contact and categorizations as varying across the four combinations of target caste and religion. Model~6 made less accurate predictions than Model~5. This shows that outgroup friendship and negative contact---but not contact quantity and positive contact---were associated with whom participants considered ``us'' and ``not us''.

\begin{figure}
\centering
\includegraphics[scale=1]{../figures/figure-4}
\caption{Estimated probability of participants categorizing a target as ``us'' versus ``not us'' as a function of the targets' group memberships (horizontal), the participants' group memberships (colour), and the reported amount of negative contact and outgroup friendship with the relevant groups (in Model~5, Table~\ref{tab:t2}). GM = General Merit, OBC = Other Backward Class, SC/ST = Scheduled Caste/Scheduled Tribe.}
\label{fig:f4}
\end{figure}

Figure~\ref{fig:f4} shows the estimated probabilities of participants categorizing targets as ``us'' as a function of their contact experiences. Across targets and participants, the odds of categorizing an outgroup target as ``us'' were $e^\beta = 1.50$, $[1.32, 1.69]$ times higher for each additional standard deviation of outgroup friendship. These odds were $e^\beta = 0.81$, $[0.72, 0.90]$ times lower for each standard deviation of negative contact. This means, for example, that GM/OBC participants who reported ``never'' having any negative contact with Muslims were more likely to categorize Indian Muslims as ``us'' than participants who reported ``sometimes'' having negative contact, \pdif{M5}{.05}{.09}{.03}. GM/OBC participants who reported no friendships with Muslims were a lot less likely to include Indian Muslims in their ingroup than participants who had 2--5 Muslim friends with whom they ``sometimes'' spent time, \pdif{M5}{.18}{.12}{.24}. Contact experiences were thus associated with whom participants categorized as ``us'' and ``not us''.

Model~7 tested whether social dominance orientation was associated with participants’ categorizations of targets from lower-status outgroups. Model~7 found little evidence for associations between participants’ categorizations and their SDO-Dominance scores ($e^\beta = 0.87$, $[0.69, 1.06]$) or SDO-Egalitarianism scores ($e^\beta = 0.98$, $[0.78, 1.21]$). Together, these findings show that group and individual differences explained whom participants included in their ingroup. As expected, past experiences with outgroup members explained why some participants included targets of (objective) caste or religions outgroups in their (subjective) ingroup, when others did not. In contrast, ideological orientations did not motivate participants to exclude lower-status groups.

\subsection{Consequences}

After examining potential antecedents, we turned to potential consequences of more inclusive social identities. We analysed how participants' categorizations related to their feeling thermometer and social distance ratings for each target in the categorization task---as well as to their responses to the intergroup threat, perceived (dis-)advantage, and policy support measures. 

\subsubsection{Intergroup attitudes}

To answer these questions, we first ran a series of multilevel multivariate regression models with participants' target-wise social distance and feeling thermometer ratings as outcome variables (Table~\ref{tab:t3}).\footnote{Linear regression models assigned weakly informative prior distributions: $\beta \sim \text{Normal}(0, 1)$ for fixed effects and $\sigma \sim \text{Half-Cauchy}(0, 1)$ for standard deviations of varying effects.} Models~0 to 3 estimated ratings (on either outcome) as varying between participants but fixed across targets (M0), as varying across participants and target categories (M1), and tested whether SC/ST participants' responses differed from GM and OBC participants' (M2), and whether OBC participants' responses differed from GM participants' (M3). Models~1 to 3 improved upon the predictions of simpler models, showing that participants' ratings depended on targets' and participants' group memberships. 

\begin{table}
\caption{Comparison of models estimating participants' social distance (SD) and feeling thermometer (FT) ratings for each target as a function of group differences and target categorizations. As in Table~\ref{tab:t2}, we selected a more complex model over a simpler model when $\frac{\Delta\textit{ELPD}}{\textit{SE}} \geq 2$. $R^2$ is a Bayesian analogue to $R^2$ in maximum likelihood estimation \protect\fullcite{gelman_rsquared_2017}.}
\centering
\figureversion{lining, tabular}
\small	
\begin{tabularx}{\linewidth}{r@{~}rXrrrrrrrr} \toprule
\# &  & Description & $R^2_\text{SD}$ & $R^2_\text{FT}$ & $\textit{ELPD}$ & $\textit{SE}$ & $\Delta\textit{ELPD}$ & $\textit{SE}$ & $\frac{\Delta\textit{ELPD}}{\textit{SE}}$ \\ \midrule 
0 &      & Participant & .23 & .26 & -16964 & 71.0 & - & - & - \\
1 & vs 0 & Category & .38 & .42 & -16338 & 79.6 & 626.8 & 40.2 & 15.6 \\
2 & vs 1 & SC/ST    & .38 & .43 & -16317 & 79.6 &  20.1 &  9.8 &  2.1 \\
3 & vs 2 & OBC      & .39 & .43 & -16293 & 80.5 &  24.2 &  9.7 &  2.5 \\ \midrule
4 & vs 3 & Categorization    & .42 & .47 & -16028 & 83.5 & 264.8 & 24.5 & 10.8 \\
5 & vs 4 & \ldots~(Category) & .42 & .47 & -16007 & 83.6 &  21.1 &  8.7 &  2.4 \\
6 & vs 5 & \ldots~(Ingroup)  & .42 & .47 & -16025 & 84.3 & -17.4 &  5.9 & -3.0 \\
\bottomrule    
\end{tabularx}
\label{tab:t3}
\end{table}

Models~4 to 6 tested whether participants who categorized a target as ``us'' rated that target more favourably than participants who categorized the same target as ``not us''. Models estimated this difference as constant across targets and participants (M4), as varying across target categories (M5), and tested whether this difference depended on participants' caste memberships (M6). Models 4 and 5, but not Model 6, made better predictions than less complex models, showing that how favourably participants felt toward a target depended on whether they had categorized that target as ``us'' or ``not us'', and that the size of this difference depended on the targets'---but not the participants'---group memberships.

\begin{figure}
\centering
\includegraphics[scale=1]{../figures/figure-5}
\caption{
Posterior probabilities of social distance ratings as a function of target categorizations (in Model~5, Table~\ref{tab:t3}). Points are the estimated mean ratings for targets categorized as ``us''; triangles are the estimated mean ratings for targets categorized as ``not us''. GM = General Merit, OBC = Other Backward Class, SC/ST = Scheduled Caste/Scheduled Tribe.
}
\label{fig:f5}
\end{figure}

\begin{figure}
\centering
\includegraphics[scale=1]{../figures/figure-6}
\caption{
Posterior probabilities of feeling thermometer ratings as a function of target categorizations (in Model~5, Table~\ref{tab:t3}). Points are the estimated mean ratings for targets categorized as ``us''; triangles are the estimated mean ratings for targets categorized as ``not us''. GM = General Merit, OBC = Other Backward Class, SC/ST = Scheduled Caste/Scheduled Tribe.
}
\label{fig:f6}
\end{figure}

Figures~\ref{fig:f5} and \ref{fig:f6} show, respectively, the estimated social distance and feeling thermometer ratings as a function of target categorizations and participants' caste memberships. For all categories, participants felt more comfortable sharing a room with targets that they categorized as ``us'' than with targets they categorized as ``not us''. This difference was smallest for Indian, Hindu, GM targets ($\beta = 0.55$, $[0.19, 0.90]$; $\text{Cohen's}~d = 0.24$, $[0.08, 0.39]$) and greatest for foreign, Muslim targets ($\beta = 1.46$, $[1.21, 1.70]$; $\text{Cohen's}~d = 0.63$, $[0.52, 0.73]$). We found a similar pattern for feeling thermometer ratings. The difference was smallest for Indian, Hindu, OBC targets ($\beta = 6.5$, $[2.1, 11.0]$, $\text{Cohen's}~d = 0.20$, $[0.06, 0.33]$) and greatest for foreign, Hindu targets ($\beta = 21.8$, $[18.6, 25.0]$, $\text{Cohen's}~d = 0.66$, $[0.56, 0.76]$). Feeling thermometer and social distance ratings were highly correlated ($r = .58$, $[.56, .60]$). Categorizing a target as ``us'' was thus associated with more warmth and less social distance toward that target.

\subsubsection{Intergroup threat}

Next, we tested whether participants' perceptions of realistic and symbolic threat depended on how inclusive their identity construals were. Results from a series of multilevel models showed that participants reported more realistic ($M = 3.62$, $[3.49, 3.75]$) than symbolic ($M = 3.21$, $[3.09, 3.33]$) threat from (same-religion) Dalits, but more symbolic ($M = 3.47$, $[3.34, 3.61]$) than realistic ($M = 3.23$, $[3.06, 3.38]$) threat from (different-religion) Muslims. Contrary to predictions, we did not find that participants felt less threatened by Muslims and Dalits if they categorized more targets from these outgroups as ``us'' (for details, see Appendix~D).

\subsubsection{Support for social change}

We investigated how participants' caste memberships shaped their perceptions of (dis-)advantages and their support for reservations. To that end, we first estimated two multivariate regression models with participants' responses to six perceived (dis-)advantage and three policy support measures as outcome variables. Model~0 estimated distinct means for each outcome, but did not consider participants' group memberships. Model~1 estimated distinct means across participants' ingroups. Model~1 made better out-of-sample predictions than Model~0, $\Delta\textit{ELPD} = 94.7$, $\textit{SE} = 14.1$, $\frac{\Delta\textit{ELPD}}{\textit{SE}} = 6.7$. This shows that participants' caste memberships shaped their perceived (dis )advantage and policy support ratings. Figures~\ref{fig:f7} and \ref{fig:f8} show the estimated group differences for the respective outcomes. Contradicting prevailing social inequalities, GM and OBC participants rated their own groups' lives as substantially harder than SC/ST members' lives. Policy support, in turn, strongly aligned with participants' caste interests. SC/ST participants supported reservations for both SC/ST and OBC students, while OBC participants only supported reservations for their own group. GM participants, for the most part, opposed reservation policies.

\begin{figure}
\centering
\includegraphics[scale=1]{../figures/figure-7}
\caption{
Posterior probabilities of perceived (dis-)advantage ratings for different target groups (right) by participants' caste ingroup (left). Diamonds mark the most plausible estimate of each mean rating; intervals encompass the 97\% most plausible estimates. GM = General Merit, OBC = Other Backward Class, SC/ST = Scheduled Caste/Scheduled Tribe.
}
\label{fig:f7}
\end{figure}

\begin{figure}
\centering
\includegraphics[scale=1]{../figures/figure-8}
\caption{
Posterior probabilities of policy support ratings for different target groups (right) by participants' caste ingroup (left). Diamonds mark the most plausible estimate of each mean rating; intervals encompass the 97\% most plausible estimates. GM = General Merit, OBC = Other Backward Class, SC/ST = Scheduled Caste/Scheduled Tribe.
}
\label{fig:f8}
\end{figure}

We then examined to what extent more inclusive identities and contact experiences accounted for individual differences in perceived (dis-)advantage and policy support ratings. To that end, we estimated a series of multilevel models with participants' responses to either the perceived (dis-)advantage or the policy support measure as outcome variable. Contrary to predictions, we did not find more inclusive identities to be associated with either perceived (dis-)advantage or policy support. Contrary to past research \cite{dixon_beyond_2012}, intergroup contact was similarly unrelated to these outcomes (for details, see Appendix~E). 

Overall, we thus found that when participants included a person in their ingroup, they had, on average, more favourable attitudes to and desired less social distance from that person. Participants' categorizations, however, were unrelated to perceptions of intergroup threat and relative (dis-)advantage, and to support for and opposition to affirmative action.

\subsection{Discussion}

This research examined how people construct their social identities from multiple cross-cutting categories, and how more inclusive identities relate to intergroup contact, outgroup attitudes, and support for social change. As hypothesized, we found that cross-group friendship was associated with more inclusive identities which, in turn, were associated with more favourable outgroup attitudes. Negative contact was associated with less inclusive identities. Contrary to past research, neither intergroup contact nor more inclusive identities were associated with perceived (dis-)advantages or support for affirmative action. Below, we discuss implications, strengths, and limitations of the present research.

Our research has implications for understanding intergroup relations in unequal societies. Among advantaged groups, our research documented patterns of inclusion and exclusion that map onto persistent social divides. Participants from dominant caste groups tended to exclude subordinate caste groups from the common ingroup; participants from the dominant Hindu religion tended to exclude Indian Muslims. Among disadvantaged groups, we found more complex patterns of inclusion and exclusion. Participants from intermediate caste groups (OBC, Other Backward Classes) faced the choice of aligning themselves with dominant caste groups (GM, General Merit), or forming a coalition with subordinate caste groups (SC/ST, Scheduled Caste/Scheduled Tribe). OBC participants tended to include dominant GM targets and exclude subordinate SC/ST targets, thus choosing derogation over coalition \cite{craig_coalition_2012}. Similarly, SC/ST participants rejected a solidarity-based social identity that includes Indian Muslims.\footnote{We did not have a sufficient number of Indian Muslim participants to examine social identification in this disadvantaged religious minority group.}

Our findings suggest that cross-group friendship can help to overcome these divisions by fostering social identities that include Indians of all castes and religions. As more inclusive identities were related to less social distance and more warmth toward caste and religious minorities, our research suggests that positive contact could help reduce interpersonal discrimination and violence against these groups. In line with recent research \cite<for example,>{hayward_toward_2017}, we found that negative contact could exacerbate social divisions by fostering less inclusive identities. More broadly, our research speaks to how contact reduces prejudice \cite{pettigrew_how_2008}. Our findings support arguments \cite{gaertner_reducing_2000, pettigrew_intergroup_1998} that contact can reduce prejudice, in part, by changing how we understand our social identities.

Our research also examined support for social change. Contrary to past research, neither positive nor negative contact \cite{hayward_how_2018, reimer_intergroup_2017} were associated with support for social change---or perceptions of relative advantage and disadvantage---in advantaged \cite{dixon_intergroup_2007} and disadvantaged \cite{dixon_beyond_2012} groups. Similarly, more inclusive identities were not associated with opposition to affirmative action among the disadvantaged \cite{dovidio_darker_2012}. Features of the participants' situation might explain this discrepancy. As university students, participants have personally experienced the impact of reservation policies. For SC/ST and OBC students, reservation policies facilitated admission to state-funded universities. This experience might explain why these students strongly support reservation (at least for their own group). For GM students, reservation policies thwarted admission to state-funded universities. This experience might explain why, in contrast to societal realities, GM students saw themselves at a disadvantage relative to other caste groups \cite<see>{norton_whites_2011}.

Our findings show that the triple crossed-categorization task \cite{dommelen_construing_2015} is an intuitive and informative method for studying social identification across multiple categories. We adapted the task to answer new questions about social identification. First, we recruited respondents from multiple groups, allowing us to study both individual and group differences \cite<see also>{brankovic_social_2015}. Second, we estimated responses as varying across targets and participants using multilevel models. This allowed more fine-grained analyses than the quantitative and qualitative summaries provided by van Dommelen and colleagues, and may explain why we found more consistent effects of intergroup contact than they did. Together, these changes open the triple crossed-categorization task to a broader range of research questions.

Our research also improves upon other approaches to studying identification across multiple social categories. For example, we could have tested our hypotheses by examining whether intergroup contact leads participants to identify with a more inclusive common ingroup (Indian) instead of their less inclusive subordinate ingroups (Hindu, Muslim). A problem with this approach is that a superordinate ingroup is more inclusive in principle, but might not be so in practice. People tend to perceive the common ingroup as more similar to their own narrow ingroup than to other outgroups \cite{wenzel_ingroup_2003}, thus excluding outgroup members from the common ingroup. We chose instead to let participants decide where to draw the line between ingroup (``us'') and outgroup (``not us'') and to thus determine the content of their subjective ingroup. Another strength of our research is that we studied intergroup relations in a context underrepresented in psychological research. Our research is one of few studies examining the effects of intergroup contact on Hindu--Muslim relations in South Asia \cite<for other examples, see>{islam_dimensions_1993, tausch_relationships_2009} and is, to our knowledge, the first study examining the effects of intergroup contact on caste relations.

Despite these strengths, our research is qualified by some methodological limitations. First, we presented all participants with the same combination of target groups. This design cannot determine whether the observed construals generalize beyond the specific combination of stimuli used. Relatedly, we did not control for factors that correlate with the categories under study, but were not made explicit. Class, rather than caste, could explain why some participants excluded targets from disadvantaged outgroups. Second, we measured categorization and attitudes for the same targets. This design cannot rule out that these variables measure the same construct, rather than represent a genuine association across constructs. One reason to think so is that intergroup threat, a more distal measure, did not correlate with participants' categorizations.\footnote{An explanation for this finding might be that threat perceptions stem from economic anxieties (e.g., fearing affirmative action as a threat to one's career) and ideological beliefs (e.g., Hindutva), rather than identity processes.} Third, we did not manipulate intergroup contact. This design cannot rule out confounding explanations of the observed relationship between contact and identification. Experimental studies tend to manipulate short-term `intergroup interactions' rather than long-term `intergroup contact' (\citeNP{macinnis_how_2015}; for rare exceptions, see \citeNP{laar_effect_2005} and \citeNP{shook_interracial_2008}). Our findings suggest, however, that cross-group friendship, not superficial interaction, is required for people to change their social identities. Future research should address these limitations by varying target categories across participants, by including more target categories, by assessing intergroup bias with proximal and distal measures, and by testing the hypothesized relationships over time.

To conclude, in a unique study of caste, religion, and nationality in South India, we found correlational evidence that intergroup contact can change not only how we see others, but also how we see ourselves---that is, that intergroup contact can foster more inclusive social identities and thus improve intergroup relations in unequal societies. Fostering more inclusive identities, however, does not seem to overcome entrenched opposition to (or undermine support for) affirmative action.

\nolinenumbers
\bibliographystyle{apacite}
\bibliography{references}

\end{document}
